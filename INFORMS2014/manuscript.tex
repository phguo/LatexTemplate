%%%%%%%%%%%%%%%%%%%%%%%%%%%%%%%%%%%%%%%%%%%%%%%%%%%%%%%%%%%%%%%%%%%%%%%%%%%%
%% Author template for Manufacturing & Service Operations Management (msom)
%% Mirko Janc, Ph.D., INFORMS, mirko.janc@informs.org
%% ver. 0.95, December 2010
%%%%%%%%%%%%%%%%%%%%%%%%%%%%%%%%%%%%%%%%%%%%%%%%%%%%%%%%%%%%%%%%%%%%%%%%%%%%
% \documentclass[msom,blindrev]{informs3} % current default for manuscript submission
\documentclass[nonblindrev]{informs3}

% \OneAndAHalfSpacedXI
% \OneAndAHalfSpacedXII % Current default line spacing
% \DoubleSpacedXII
% \DoubleSpacedXI

% If hyperref is used, dvi-to-ps driver of choice must be declared as
%   an additional option to the \documentclass. For example
%\documentclass[dvips,msom]{informs3}      % if dvips is used
%\documentclass[dvipsone,msom]{informs3}   % if dvipsone is used, etc.

% Private macros here (check that there is no clash with the style)

% Natbib setup for author-year style
\usepackage{natbib}
 \bibpunct[, ]{(}{)}{,}{a}{}{,}%
 \def\bibfont{\footnotesize}%
 \def\bibsep{\smallskipamount}%
 \def\bibhang{20pt}%
 \def\newblock{\ }%
 \def\BIBand{and}%

%% Setup of theorem styles. Outcomment only one. 
%% Preferred default is the first option.
\TheoremsNumberedThrough     % Preferred (Theorem 1, Lemma 1, Theorem 2)
%\TheoremsNumberedByChapter  % (Theorem 1.1, Lema 1.1, Theorem 1.2)

%% Setup of the equation numbering system. Outcomment only one.
%% Preferred default is the first option.
\EquationsNumberedThrough    % Default: (1), (2), ...
%\EquationsNumberedBySection % (1.1), (1.2), ...

% In the reviewing and copyediting stage enter the manuscript number.
\MANUSCRIPTNO{not assigned} % When the article is logged in and DOI assigned to it,
                 %   this manuscript number is no longer necessary








% <<<<< Added by guoph start <<<<< %

% Chinese
% \usepackage{fontspec} 
% \usepackage[slantfont, boldfont]{xeCJK}
% \setCJKmainfont{SimSong-Regular}
% \XeTeXlinebreaklocale "zh"
% \XeTeXlinebreakskip = 0pt plus 1pt

% \usepackage{lmodern}

% Palatino font 
% \usepackage{mathpazo}
% \usepackage{microtype}
% Tweak caption labels 
% \usepackage[margin=15pt,font=small,labelfont={bf,sf}]{caption} 
% section bold font
% \usepackage{sectsty}

% line number
% \usepackage{lineno}
% \modulolinenumbers[1]
% \linenumbers

\usepackage{amsmath,graphicx,bm,amsfonts,mathrsfs}
\usepackage[short]{optidef} 
% \usepackage{amsthm}
% \newtheorem{theorem}{Theorem}
% \newtheorem{property}{Property}
% \newtheorem{example}{Example}
% % \newtheorem{proof}{Proof}[section]
% \newenvironment{proof}{{\indent \indent \it Proof:\quad}}{\hfill $\blacksquare$\par}
\newcommand*{\logeq}{\ratio\Leftrightarrow}

\DeclareMathOperator*{\argminA}{argmin} % Jan Hlavacek

% \sout{}
% \usepackage{ulem} 

% \ul{}
\usepackage{soul}
 
% appendix
% \usepackage[title]{appendix} 
% \def\appendixname{Appendix}
 
% Table
\usepackage{booktabs} 
\usepackage[flushleft]{threeparttable} 
\usepackage{tabularx, xltabular, multirow}
% display break for "align" environment
\allowdisplaybreaks[1] 

% Algorithm
\usepackage[linesnumbered,ruled,vlined]{algorithm2e} 
\usepackage{setspace}
% \AtBeginEnvironment{algorithm}{\setstretch{1.2}}
% algorithm highlight
% \newcommand\mycommfont[1]{\ttfamily\textcolor{NavyBlue}{#1}}
\SetCommentSty{mycommfont}

% For \upcite{}
% \newcommand{\upcite}[1]{\textsuperscript{\textsuperscript{\cite{#1}}}}

% Handle "Package hyperref Warning: Token not allowed in a PDF string."
% \hypersetup{pdfauthor={guoph}}

% Row spacing
% \renewcommand{\baselinestretch}{1.4}

% Handle "elsarticle Package inputenc Error: Invalid UTF-8 byte sequence."
% When submit, comment this line and copy the content of .bbl file the the end but before \end{document} of this file.
% \UseRawInputEncoding

% TODO
\newcommand\todo[1]{{\textcolor{violet}{\bf [\ul{TODO}: #1]}}}
% \renewcommand\todo[1]{}
% comment
\newcommand\cmt[1]{{\textcolor{NavyBlue}{[#1]}}} 
% \renewcommand\cmt[1]{}
% highlight
\definecolor{internationalorange}{rgb}{1.0, 0.31, 0.0}
\newcommand\hlt[1]{{\textcolor{internationalorange}{#1}}}
% \renewcommand\hlt[1]{{\color{black}{#1}}}

% comment
\usepackage{comment}

\usepackage{subfigure} 

\usepackage{makecell}

% for MidnightBlue and NavyBlue color
\usepackage[dvipsnames]{xcolor}
\usepackage[colorlinks=True,allcolors=NavyBlue]{hyperref} 
\usepackage[nameinlink]{cleveref}
\Crefformat{equation}{(#2#1#3)}

% TikZ
\usepackage{pgfplots}
\usepackage{tikz}
\usetikzlibrary{decorations.pathreplacing}
\usetikzlibrary{fit}  % fitting shapes to coordinates
\usetikzlibrary{backgrounds}  % drawing the background after the foreground

\usepgfplotslibrary{fillbetween}
\definecolor{RoyalAzure}{rgb}{0.0, 0.22, 0.66}  
\pgfplotsset{compat=1.16}

% \usepackage{makecell}
\usepackage{diagbox}


\usepackage{dcolumn}
\usepackage{lscape}
\usepackage{verbatim}
 
\usepackage{marginnote}

% \usepackage{fancyhdr}
% \pagestyle{fancy}
% \fancyhead{}
% \fancyhead[L]{{\bf Guo \& Zhu:} {\it Stochastic Decentralized Dual Dynamic Programming - S3DP}}
% \fancyhead[R]{\thepage}
% \fancyfoot{}

\usepackage{cancel} 

\usepackage{endnotes}
\let\footnote=\endnote

% move blank on each page to the bottom
\raggedbottom


% set page size
\usepackage{geometry}
\geometry{
        a4paper,
        textheight=250mm, 
        textwidth=162mm
}

% ###### INFORMS Template customization start ###### %

% remove INFORMS disclaimer at the top of title page
\renewcommand{\theARTICLETOP}{}

% remove Article submitted to {\it\theJOURNAL}; manuscript no. {\theMANUSCRIPTNO}}
% \RRHSecondLine{}
% \LRHSecondLine{}
\RRHSecondLine{Submmited to: ***; Manuscript ID: ***}
\LRHSecondLine{Submmited to: ***; Manuscript ID: ***}


% bibliographystyle
% !!! issn is removed for article in informs2014.bst
\bibliographystyle{informs2014}

% ###### INFORMS Template customization end   ###### %


% >>>>> Added by guoph end >>>>> %











%%%%%%%%%%%%%%%%
\begin{document}
%%%%%%%%%%%%%%%%

% Outcomment only when entries are known. Otherwise leave as is and 
%   default values will be used.
%\setcounter{page}{1}
%\VOLUME{00}%
%\NO{0}%
%\MONTH{Xxxxx}% (month or a similar seasonal id)
%\YEAR{0000}% e.g., 2005
%\FIRSTPAGE{000}%
%\LASTPAGE{000}%
%\SHORTYEAR{00}% shortened year (two-digit)
%\ISSUE{0000} %
%\LONGFIRSTPAGE{0001} %
%\DOI{10.1287/xxxx.0000.0000}%

% Author's names for the running heads
% Sample depending on the number of authors;
% \RUNAUTHOR{Jones}
% \RUNAUTHOR{Jones and Wilson}
% \RUNAUTHOR{Jones, Miller, and Wilson}
% \RUNAUTHOR{Jones et al.} % for four or more authors
% Enter authors following the given pattern:
\RUNAUTHOR{Authors}

% Title or shortened title suitable for running heads. Sample:
% \RUNTITLE{Bundling Information Goods of Decreasing Value}
% Enter the (shortened) title:
\RUNTITLE{Short titile}

% Full title. Sample:
% \TITLE{Bundling Information Goods of Decreasing Value}
% Enter the full title:
\TITLE{Title}

% Block of authors and their affiliations starts here:
% NOTE: Authors with same affiliation, if the order of authors allows, 
%   should be entered in ONE field, separated by a comma. 
%   \EMAIL field can be repeated if more than one author
\ARTICLEAUTHORS{%
\AUTHOR{A1, A1\thanks{Corresponding author. \hfill \it \today}}
\AFF{College of ***, *** University, ** city zipcode, China, \{\EMAIL{a@a.com, b@b.com}\}}
% Enter all authors
} % end of the block

\ABSTRACT{%
% Text of your abstract % Enter your abstract
Abstract here.
}%

% Sample
%\KEYWORDS{deterministic inventory theory; infinite linear programming duality; 
%  existence of optimal policies; semi-Markov decision process; cyclic schedule}

% Fill in data. If unknown, outcomment the field
\KEYWORDS{keywords here}
% \HISTORY{}

\maketitle
%%%%%%%%%%%%%%%%%%%%%%%%%%%%%%%%%%%%%%%%%%%%%%%%%%%%%%%%%%%%%%%%%%%%%%

% Samples of sectioning (and labeling) in MSOM
% NOTE: (1) \section and \subsection do NOT end with a period
%       (2) \subsubsection and lower need end punctuation
%       (3) capitalization is as shown (title style).
%
%\section{Introduction.}\label{intro} %%1.
%\subsection{Duality and the Classical EOQ Problem.}\label{class-EOQ} %% 1.1.
%\subsection{Outline.}\label{outline1} %% 1.2.
%\subsubsection{Cyclic Schedules for the General Deterministic SMDP.}
%  \label{cyclic-schedules} %% 1.2.1
%\section{Problem Description.}\label{problemdescription} %% 2.

% Text of your paper here





\section{Introduction} \label{SEC: Introduction}


The old (2014) INFORMS Template.

\cite{acimovicModelsMetricsAssess2016}



 
% \theendnotes

% Acknowledgments here
\ACKNOWLEDGMENT{%
This work was supported by ***
}% Leave this (end of acknowledgment)


% Appendix here
% Options are (1) APPENDIX (with or without general title) or 
%             (2) APPENDICES (if it has more than one unrelated sections)
% Outcomment the appropriate case if necessary

% \begin{APPENDIX}{<Title of the Appendix>}
% \end{APPENDIX}
%   or
% \begin{APPENDICES}
% % \section{<Title of Section A>}
% % \section{<Title of Section B>}
% % etc
% \end{APPENDICES}


% References here (outcomment the appropriate case) 

% CASE 1: BiBTeX used to constantly update the references 
%   (while the paper is being written).
%\bibliographystyle{informs2014} % outcomment this and next line in Case 1
%\bibliography{<your bib file(s)>} % if more than one, comma separated

% CASE 2: BiBTeX used to generate mypaper.bbl (to be further fine tuned)
%\input{mypaper.bbl} % outcomment this line in Case 2


\bibliography{bibfile}

% \newpage

% \appendix
%%%%%%%%%% Merge with supplemental materials %%%%%%%%%%
% \widetext
\clearpage
\begin{center}
\textbf{Supplemental Material}
\end{center}
%%%%%%%%%% Merge with supplemental materials %%%%%%%%%%
%%%%%%%%%% Prefix a "S" to all equations, figures, tables and reset the counter %%%%%%%%%%
\setcounter{section}{0}
\setcounter{equation}{0}
\setcounter{figure}{0}
\setcounter{table}{0}
\setcounter{page}{1}
\makeatletter
\renewcommand{\theequation}{S\arabic{equation}}
\renewcommand{\thefigure}{S\arabic{figure}}
\renewcommand{\thetable}{S\arabic{table}}
\renewcommand{\bibnumfmt}[1]{[S#1]}
\renewcommand{\citenumfont}[1]{S#1}
\renewcommand{\thesection}{S\arabic{section}}
\renewcommand{\thesubsection}{S.\arabic{subsection}}
%%%%%%%%%% Prefix a "S" to all equations, figures, tables and reset the counter %%%%%%%%%%


\subsection{S1} \label{SEC: S1}

 



\end{document}


