\iffalse
  % 本块代码被上方的 iffalse 注释掉,如需使用,请改为 iftrue
  % 使用 Noto 字体替换中文宋体、黑体
  \setCJKfamilyfont{\CJKrmdefault}[BoldFont=Noto Serif CJK SC Bold]{Noto Serif CJK SC}
  \renewcommand\songti{\CJKfamily{\CJKrmdefault}}
  \setCJKfamilyfont{\CJKsfdefault}[BoldFont=Noto Sans CJK SC Bold]{Noto Sans CJK SC Medium}
  \renewcommand\heiti{\CJKfamily{\CJKsfdefault}}
\fi

\iffalse
  % 本块代码被上方的 iffalse 注释掉,如需使用,请改为 iftrue
  % 在 XeLaTeX + ctexbook 环境下使用 Noto 日文字体
  \setCJKfamilyfont{mc}[BoldFont=Noto Serif CJK JP Bold]{Noto Serif CJK JP}
  \newcommand\mcfamily{\CJKfamily{mc}}
  \setCJKfamilyfont{gt}[BoldFont=Noto Sans CJK JP Bold]{Noto Sans CJK JP}
  \newcommand\gtfamily{\CJKfamily{gt}}
\fi


% 设置基本文档信息,\linebreak 前面不要有空格,否则在无需换行的场合,中文之间的空格无法消除
\nuaaset{
  thesisid = {},   % 论文编号
  title = {},
  author = {},
  college = {经济与管理学院},
  advisers = {*** \ 教授},
  % applydate = {二〇一八年六月}  % 默认当前日期
  %
  % 本科
  % major = {\LaTeX{} 科学与技术},
  % studentid = {131810299},
  % classid = {应用技术},           % 班级的名称
  % industrialadvisers = {Jack Ma}, % 企业导师,若无请删除或注释本行
  % 硕/博士
  majorsubject = {管理科学与工程},
  researchfield = {},
  libraryclassid = {},       % 中图分类号
  subjectclassid = {},      % 学科分类号
}
\nuaasetEn{
  title = {},
  author = {},
  college = {College of Economics and Management},
  majorsubject = {Management Science and Engineering},
  advisers = {Prof. \ ***},
  degreefull = {Doctor of Philosophy},
  % applydate = {June, 8012}
}

% 摘要
\begin{abstract}
中文摘要

\vspace*{-2em}
\end{abstract}
\keywords{关键词}

\begin{abstractEn}
Abstract

\vspace*{-2em}
\end{abstractEn}
\keywordsEn{keywords}


% 请按自己的论文排版需求,随意修改以下全局设置

% \usepackage{subfig}
\usepackage{subfigure}
% \usepackage{longtable}
\usepackage{tabularx, xltabular}

% above is added by guoph

\usepackage{rotating}
\usepackage[usenames,dvipsnames]{xcolor}
\usepackage{tikz}
\usepackage{pgfplots}
\pgfplotsset{compat=1.16}
\pgfplotsset{
  table/search path={./fig/},
}
\usepackage{ifthen}
\usepackage{siunitx}
\usepackage{listings}
\usepackage{multirow}
\usepackage{pifont}

\lstdefinestyle{lstStyleBase}{%
  basicstyle=\small\ttfamily,
  aboveskip=\medskipamount,
  belowskip=\medskipamount,
  lineskip=0pt,
  boxpos=c,
  showlines=false,
  extendedchars=true,
  upquote=true,
  tabsize=2,
  showtabs=false,
  showspaces=false,
  showstringspaces=false,
  numbers=left,
  numberstyle=\footnotesize,
  linewidth=\linewidth,
  xleftmargin=\parindent,
  xrightmargin=0pt,
  resetmargins=false,
  breaklines=true,
  breakatwhitespace=false,
  breakindent=0pt,
  breakautoindent=true,
  columns=flexible,
  keepspaces=true,
  framesep=3pt,
  rulesep=2pt,
  framerule=1pt,
  backgroundcolor=\color{gray!5},
  stringstyle=\color{green!40!black!100},
  keywordstyle=\bfseries\color{blue!50!black},
  commentstyle=\slshape\color{black!60}}

%\usetikzlibrary{external}
%\tikzexternalize % activate!

\newcommand\cs[1]{\texttt{\textbackslash#1}}
\newcommand\pkg[1]{\texttt{#1}\textsuperscript{PKG}}
\newcommand\env[1]{\texttt{#1}}

\theoremstyle{nuaaplain}
\nuaatheoremchapu{definition}{定义}
\nuaatheoremchapu{assumption}{假设}
\nuaatheoremchap{exercise}{练习}
% \nuaatheoremchap{nonsense}{胡诌}
\nuaatheoremg[句]{lines}{句子}

% <<<<< Added by guoph start <<<<< %

\usepackage{amsmath,graphicx,bm,amsfonts,mathrsfs}
\usepackage[short]{optidef}
\usepackage{amsthm}
\newtheorem{theorem}{定理}
\newtheorem{property}{Property}
\newtheorem{example}{Example}
% % \newtheorem{proof}{Proof}[section]
% \newenvironment{proof}{{\indent \indent \it Proof:\quad}}{\hfill $\blacksquare$\par}
\newcommand*{\logeq}{\ratio\Leftrightarrow}
 
% % appendix
% % \usepackage[title]{appendix}
% % \def\appendixname{Appendix}

% Table
\usepackage{booktabs} 
\usepackage[flushleft]{threeparttable} 
\usepackage{tabularx, xltabular, multirow}
% display break for "align" environment
\allowdisplaybreaks[1]

% Algorithm
\usepackage[linesnumbered,ruled,vlined]{algorithm2e}
\usepackage{setspace}
% \AtBeginEnvironment{algorithm}{\setstretch{1.2}}
% algorithm highlight
% \newcommand\mycommfont[1]{\ttfamily\textcolor{NavyBlue}{#1}}
\SetCommentSty{mycommfont}

% % \usepackage{subfigure} 

\usepackage{makecell}

% % TikZ
% \usepackage{pgfplots}
% \usepackage{tikz}
% \usetikzlibrary{decorations.pathreplacing}
% \usetikzlibrary{fit}  % fitting shapes to coordinates
% \usetikzlibrary{backgrounds}  % drawing the background after the foreground

% \usepgfplotslibrary{fillbetween}
% \definecolor{RoyalAzure}{rgb}{0.0, 0.22, 0.66}
% \pgfplotsset{compat=1.16}

\usepackage{diagbox}
\usepackage{dcolumn}
\usepackage{lscape}
\usepackage{verbatim}

% \usepackage{endnotes}
% \let\footnote=\endnote

% move blank on each page to the bottom
\raggedbottom

% 中文算法环境
\renewcommand{\algorithmcfname}{算法}
\renewcommand{\KwData}{{输入:}}
\renewcommand{\KwResult}{{\\输出:}}

% highlight
\definecolor{internationalorange}{rgb}{1.0, 0.31, 0.0}
\newcommand\hlt[1]{{\color{black}#1}}

% % >>>>> Added by guoph end >>>>> %
