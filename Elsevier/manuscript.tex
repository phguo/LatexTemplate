\documentclass[3p,review,a4paper]{elsarticle}
\journal{Elsevier}

% <<<<< Added by guoph start <<<<< %

% set margin
\geometry{
        textheight=24cm, 
        % textwidth=17cm
        % a4paper,
}

% Chinese
% \usepackage{fontspec}
% \usepackage[slantfont, boldfont]{xeCJK}
% \setCJKmainfont{SimSong-Regular}
% \XeTeXlinebreaklocale "zh"
% \XeTeXlinebreakskip = 0pt plus 1pt

% Palatino font
% \usepackage{mathpazo}
% \usepackage{microtype}
% Tweak caption labels
% \usepackage[margin=15pt,font=small,labelfont={bf,sf}]{caption}
% section bold font
% \usepackage{sectsty}

% line number
% \usepackage{lineno}
% \modulolinenumbers[1]
% \linenumbers

\usepackage{amsmath,graphicx,bm,amsfonts,mathrsfs}
\usepackage[short]{optidef}
\usepackage{amsthm}
\newtheorem{theorem}{Theorem}
% % \newtheorem{proof}{Proof}[section]
% \newenvironment{proof}{{\indent \indent \it Proof:\quad}}{\hfill $\blacksquare$\par}
\newcommand*{\logeq}{\ratio\Leftrightarrow}

% \sout{}
% \usepackage{ulem}

% appendix
% \usepackage[title]{appendix}
% \def\appendixname{Appendix}

% Table
\usepackage{booktabs} 
\usepackage[flushleft]{threeparttable} 
\usepackage{tabularx, xltabular, multirow}
% display break for "align" environment
\allowdisplaybreaks[1]

% Algorithm
\usepackage[linesnumbered,ruled,vlined]{algorithm2e}
\usepackage{setspace}
% \AtBeginEnvironment{algorithm}{\setstretch{200}}
% algorithm highlight
% \newcommand\mycommfont[1]{\ttfamily\textcolor{NavyBlue}{#1}}
\SetCommentSty{mycommfont}

% For \upcite{}
% \newcommand{\upcite}[1]{\textsuperscript{\textsuperscript{\cite{#1}}}}

% Handle "Package hyperref Warning: Token not allowed in a PDF string."
% \hypersetup{pdfauthor={guoph}}

% Row spacing
% \renewcommand{\baselinestretch}{1.4}

% Handle "elsarticle Package inputenc Error: Invalid UTF-8 byte sequence."
% When submit, comment this line and copy the content of .bbl file the the end but before \end{document} of this file.
\UseRawInputEncoding

% TODO
\newcommand\todo[1]{{\textcolor{NavyBlue}{\sc TODO: (#1)}}}
% \renewcommand\todo[1]{}
% comment
\newcommand\cmt[1]{{\textcolor{NavyBlue}{[#1]}}} 
% \renewcommand\cmt[1]{}
% highlight
\newcommand\highlight[1]{{\color{NavyBlue}{#1}}}
% \newcommand\highlight[1]{{\textcolor{black}{#1}}}

% comment
\usepackage{comment}

\usepackage{subfigure} 

\usepackage{makecell}

% for MidnightBlue and NavyBlue color
\usepackage[dvipsnames]{xcolor}
\usepackage[colorlinks=True,allcolors=NavyBlue]{hyperref} 
\usepackage[nameinlink]{cleveref}
\crefformat{equation}{(#2#1#3)} 

% >>>>> Added by guoph end >>>>> %



%%%%%%%%%%%%%%%%%%%%%%%
%% Elsevier bibliography styles
%%%%%%%%%%%%%%%%%%%%%%%
%% To change the style, put a % in front of the second line of the current style and,
%% remove the % from the second line of the style you would like to use.
%%%%%%%%%%%%%%%%%%%%%%%
%% Numbered
% \bibliographystyle{bib_style/model1-num-names}
%% Numbered without titles
% \bibliographystyle{bib_style/model1a-num-names}
%% Harvard
% \bibliographystyle{bib_style/model2-names.bst}\biboptions{authoryear}
%% Vancouver numbered,
%\usepackage{bib_style/numcompress}\bibliographystyle{bib_style/model3-num-names}
%% Vancouver name/year
% \usepackage{bib_style/numcompress}\bibliographystyle{bib_style/model4-names}\biboptions{authoryear}
%% APA style
\bibliographystyle{bib_style/model5-names.bst}\biboptions{authoryear}
%% AMA style
% \usepackage{bib_style/numcompress}\bibliographystyle{bib_style/model6-num-names}
%% `Elsevier LaTeX' style
% \bibliographystyle{bib_style/elsarticle-num}
%%%%%%%%%%%%%%%%%%%%%%%



\begin{document}

\hypersetup{allcolors=NavyBlue}  
% \fontsize{10}{12}\selectfont % default
% \fontsize{10}{11}\selectfont

\begin{frontmatter}

\title{This is a \LaTeX template for Elsevier}

% (or branch and check)
%% Group authors per affiliation:
% \author{Elsevier\fnref{myfootnote}}
% \address{Radarweg 29, Amsterdam}
% \fntext[myfootnote]{Since 1880.}

%% or include affiliations in footnotes:
\author[nuaa]{Penghui Guo
\corref{correspondingauthor}
}
\cortext[correspondingauthor]{Corresponding author}
\ead{m@guo.ph}

\address[nuaa]{College of Economics and Management, Nanjing University of Aeronautics and Astronautics, Nanjing 211106, China}


\begin{abstract}
        Abstract
\end{abstract}

\begin{keyword}
        input keyword here
        \sep input another keyword here
\end{keyword}

\end{frontmatter}

% \linenumbers

\newpage

\section{Introduction} \label{SEC: Introduction}

\todo{Something to do}

\highlight{Something be highlighted}

\cmt{Some comment}

invisible comment:
\begin{comment}
        invisible comment
\end{comment}

\section{Literature review} \label{SEC: Related works}
 
\subsection{Literature review subsection} \label{SUBSEC: Literature review subsection}

citation: \cite{grassTwostageStochasticProgramming2016},

citation in bracket: \citep{grassTwostageStochasticProgramming2016}, 

\section{Mathematical model} \label{SEC: Mathmatical models}

\begin{center}
        \footnotesize
        \def\arraystretch{.8}
\begin{xltabular}{.9\textwidth}{lX}
\caption{Notations} \label{TAB: notations} \\
        \toprule 
        % ====================
        % ======= Sets =======
        % ====================
        \multicolumn{2}{l}{\bf Sets:} \\
        % \midrule
        \(T\) & blablabla \\
        \midrule
        % ====================
        % ==== Parameters ====
        % ====================
        \multicolumn{2}{l}{\bf Parameters:} \\
        % \midrule
        \(M\) & blablabla \\
        \midrule
        % ====================
        % === Decision Var ===
        % ====================
        \multicolumn{2}{l}{\bf First-stage decision variables:} \\
        % \midrule
         \(y_{ik}\) & blablabla \\
        \midrule 
        \multicolumn{2}{l}{\bf Second-stage decision variables:} \\
        % \midrule
         \(t_{ij}^s\) & blablabla \\
        \midrule
        \multicolumn{2}{l}{\bf Auxiliary decision variables:} \\
        % \midrule
         \(x_{pq}^s\) & blablabla \\
        \bottomrule
\end{xltabular}
\end{center}

blablabla

Model \cref{MODEL SPRP}

\begin{subequations} \label{MODEL SPRP}
        \renewcommand{\theequation}{\theparentequation.\arabic{equation}}
\begin{align} 
\label{cost objective}
\begin{split}
        \text{[SPRP] ~minimize~}
        & \sum_{i \in T} \sum_{k \in K} c^k y_{ik}
        + \sum_{i \in T} q l_i
        + \sum_{i \in T} \sum_{h \in {H}} c_1^h w_i^h \\
        & + \mathbb{E}_{s \in S} \left(\sum_{i \in T}\mathcal{R}_{is}\right)
\end{split} \\
% # # # # # # # # first-stage constraints # # # # # # # # 
\label{constraint}
\text{subject to~}
        & \sum_{k \in K} y_{ik} \leq 1, 
        & \forall i \in T
\end{align}
\end{subequations}
% # # # # # # # # explain objetive function and original constraint # # # # # # # # 
\eqref{cost objective} blablabla.


\section{Solution method} \label{SEC: Solution method}

\subsection{Solution method subsection} \label{SUBSEC: Solution method subsection}

\subsubsection{Solution method subsubsection} \label{SUBSUBSEC: Solution method subsubsection}

\section{Numerical experiments}\label{SEC: Numerical experiments}

% \begin{table}[!htp]
\begin{table}[!t]
        \footnotesize
        \centering
        \def\arraystretch{1.1}
\caption{Table}
% \resizebox{.9\textwidth}{!}{
\begin{threeparttable}
\begin{tabular}{cll}
        \toprule 
        Parameters & Description & Value \\
        \midrule
        \multicolumn{2}{l}{\bf Variable parameters:} \\
        \(|T|\) & blablabla & blablabla \\
        \midrule
        \multicolumn{2}{l}{\bf Deterministic parameters:} \\
        \(p_{j}\) & blablabla & blablabla \\
        \bottomrule
\end{tabular}
\begin{tablenotes}
\item [1] balblabla
\end{tablenotes}
\end{threeparttable}
% }
\label{TAB: instance}
\end{table} 


\begin{table}[!htp]
        \def\arraystretch{1.1}
        \footnotesize
        \centering
\caption{Table}
\resizebox{1\textwidth}{!}{
\begin{threeparttable}
\begin{tabular}{crrrrrrrrrrrr}
\toprule
        \multirow{2}{*}{Instance} & \multicolumn{3}{l}{A} & \multicolumn{3}{l}{B} & \multicolumn{3}{l}{C} & \multicolumn{3}{l}{D}\\
        \cmidrule(rrrr){2-4} \cmidrule(rrrr){5-7} \cmidrule(rrrr){8-10} \cmidrule(rrrr){11-13} 
                & Obj. & \makecell[c]{Gap\\(\%)} & \makecell[c]{Time\\(s)} & Obj. & \makecell[c]{Gap\\(\%)} & \makecell[c]{Time\\(s)} & Obj. & \makecell[c]{Gap'\\(\%)} & \makecell[c]{Time\\(s)} & Obj. & \makecell[c]{Gap\\(\%)} & \makecell[c]{Time\\(s)}\\
\midrule
**-**-** & {\bf *000} & 0.00 & 000 & {\bf *000} & 0.00 & 000 & {\bf *000} & 0.00 & 000 & {\bf *000} & 0.00 & 000 \\
\bottomrule
\end{tabular}
\begin{tablenotes}
        \item blablabla
        \end{tablenotes}
\end{threeparttable}
}
\label{TAB: results}
\end{table}


\begin{figure}[!htp]
        \centering
        \subfigure[Sub figure]{
                \includegraphics[width=.30\textwidth]{figure/blanck.png}
                \label{FIG: a}
                }
        \subfigure[Sub figure]{
                \includegraphics[width=.30\textwidth]{figure/blanck.png}
                \label{FIG: b}
                }
                \begin{minipage}[b]{.235\textwidth}
                        \centering
                        \subfigure[Sub figure]{
                                \includegraphics[width=\textwidth]{figure/blanck.png}
                                \label{FIG: c}
                                }
                        \subfigure[Sub figure]{
                                \includegraphics[width=\textwidth]{figure/blanck.png}
                                \label{FIG: d}
                                }
                \end{minipage}
        \caption{A figure 
        }
        \label{FIG: all}
\end{figure}

reference \Cref{FIG: all}, \Cref{FIG: a}, \Cref{FIG: b}, \Cref{FIG: c}, \Cref{FIG: d}.

another reference \cref{FIG: all}, \cref{FIG: a}, \cref{FIG: b}, \cref{FIG: c}, \cref{FIG: d}.

\ref{POOF: name}, \cref{THEOREM: name}


\section{Acknowledgment}

This work was supported by 
the United States National Science Foundation (NSF) [grant number ***]; 
the National Natural Science Foundation of China (NSFC) [grant number ***]; 
and the China Scholarship Council (CSC) [grant number ***].
We also appreciate the anonymous reviewers for their valuable suggestions.

\appendix

\section{Some proof} \label{POOF: name}

\begin{theorem} \label{THEOREM: name}
        Some theorem.
\end{theorem}

\begin{proof}        
        \begin{align}
                \text{Equations}
        \end{align}
        \begin{align*}
                \text{Equations without number}
        \end{align*}
        \begin{enumerate}[\it\text{Case} 1:]
                \item Case
                \item \label{case 2} 
                \begin{enumerate}[\it\text{Sub case} (1):]
                        \item Sub case
                \end{enumerate}
        \end{enumerate}
        proof is end here
\end{proof}

\section{Pseudo code} \label{Pseudo code}
 
% \IncMargin{.3em}
\begin{algorithm}
        \setstretch{1}
        % \small
        \footnotesize
        % \SetStartEndCondition{ }{}{}
        \SetKwProg{Fn}{def}{\string:}{}
        \SetKwFunction{Benders}{Benders}
        \SetKwFunction{NBenders}{function\_name}
        % \SetKwFunction{Range}{range}
        % \SetKw{KwTo}{in}
        \SetKwFor{For}{for}{\string:}{}
        \SetKwFor{ForAll}{for}{\string:}{}
        \SetKwIF{If}{ElseIf}{Else}{if}{:}{elif}{else:}{}
        \SetKwFor{While}{while}{:}{}
        \AlgoDontDisplayBlockMarkers
        % \SetAlgoNoEnd
        % \SetAlgoNoLine
        \SetSideCommentRight
        \SetKwData{True}{True}
        \SetKwData{False}{False}
        \SetKwData{None}{None}

\caption{This is an algorithm} \label{A: algorithm}
\KwData{data}
\KwResult{results}
% \BlankLine
\Fn{\NBenders{PMP\(_u\)}}{
        gap \(\leftarrow +\infty\)\;
        \While{gap \(\neq 0\)}{
                do something\;
        \If{condition \(>\) value}{
                something = \True\;
                \ForAll{\(i \in T, s \in S\)}{
                        do something\; \label{A2: label for this line}
                }
                \If{ondition \(>\) value}{
                        a \(\leftarrow\) b
                        \tcp*[r]{blablabla}
                }
        }
        \Else(\tcp*[f]{Return something}){
                \KwRet{something}\; 
        }
        \(v \leftarrow v + 1\)\;
        }
        \KwRet{something}
}
\end{algorithm}

% \DecMargin{.3em}

\bibliography{mybibfile}
 
\end{document}